% 导入包
\usepackage{hologo}  % LOGO
%%% 数学公式
\usepackage{amsmath,amsxtra}  % AMS
\usepackage{bm}  % 加粗符号
\usepackage{cases}  % 行编号数学环境
\usepackage{extarrows}  % 可延长箭头
\usepackage{mathtools}  % 数学公式
\usepackage{mhchem}  % 化学公式包  \ce
\usepackage{ntheorem}  % 获取数学模式的内容
\usepackage{upgreek}  % 直立希腊字母
\usepackage{xfrac}  % 行内除法 \sfrac
%%% 环境设置
\usepackage{caption} % 标题
\usepackage{subcaption}  % 子标题
\usepackage{float}  % 浮动控制
\usepackage{wrapfig}  % 环绕排版
%%% 表格
\usepackage{multirow,multicol}  % 跨行、跨列
\usepackage{rotating,makecell}  % 控制表头
\usepackage{diagbox}  % 表头斜线
\usepackage{array,tabularx}  % 定宽表
\usepackage{longtable}  % 长表
\usepackage{threeparttable}  % 三线表
%%% 图片
\usepackage{graphicx}  % 插图
\usepackage{lscape}  % 页面旋转
%%% 定理
\usepackage{enumerate}  % 列表格式, P105
\usepackage{theorem}  % 定理格式, P107
%%% 算法
\usepackage[ruled]{algorithm2e}
%%% 参考文献
\usepackage[style=ieee,backend=biber,natbib=true]{biblatex}
%%% 绘图
\usepackage[all,pdf]{xy}
\usepackage{tikz}
\usepackage{siunitx}
%%% 字体
\usepackage[math-style=ISO]{unicode-math}  % 数学字体
\usepackage{fontspec}  % 字体设置
\usepackage{scalefnt} % 字体缩放
\usepackage{fontawesome5}
%%% 基础格式
\usepackage{ctex,xeCJK}  % 中文环境
\usepackage{fancyhdr}  % 页眉页脚
\usepackage{geometry}  % 页边距
\usepackage{tocloft}  % 目录格式
%%% 字体格式: rm,sf,tt|bf(default),up,it,sl,sc
%%% 字号: big(default),medium,small,tiny
%%% 对齐方式: raggedright(default),center,raggedleft
\usepackage[raggedright]{titlesec}  % 章节格式
\usepackage{chngcntr}  % 取消章节关联
\usepackage{ragged2e}  % 段落格式
\usepackage{shapepar}  % 段落形状
\usepackage{nameref}  % 名称引用
\usepackage{color,xcolor}  % 色彩
\usepackage{hyperref}  % 链接、信息
% \usepackage{tocbibind}  % 为目录、参考文献和索引生成目录项
% \usepackage{bookmark} % 书签
% \usepackage{layout}  % 显示当前页面布局: \layout{}
% \usepackage{showframe}  % 标记出页面布局





% 定义数学符号
\DeclareMathOperator{\dif}{d\!}  % 积分d符号
\newcommand{\degree}{^\circ}  % 角度
\newcommand{\abs}[1]{\lvert#1\rvert}





% 字体设置
%%% 数学字体
\setmathfont{STIX Two Math}
% \setmathfont{Asana Math}
% \setmathfont{XITS Math}
% \setmathfont{TeX Gyre Schola Math}
% \setmathfont{TeX Gyre Pagella Math}
% \setmathfont{TeX Gyre Termes Math}
%%% 西文字体
\setmainfont{Source Sans Pro}
% \setmainfont{Source Serif Pro}
% \setmainfont{Source Code Pro}
% \setmainfont{Fira Code}
% \setmainfont{Times New Roman}
% \setmainfont{Courier New}
%%% 中文字体
% \setCJKmainfont[BoldFont=SimHei,ItalicFont=KaiTi]{SimSun} % 中易
% \setCJKmainfont[BoldFont=FZHei-B01,ItalicFont=FZKai-Z03]{FZShuSong-Z01}  % 方正
\setCJKmainfont[BoldFont=Source Han Sans SC]{Source Han Serif SC}  % 思源
% \setCJKmainfont[BoldFont=FandolHei,ItalicFont=FandolKai]{FandolSong}  % Fandol





% 字号设置
\newcommand{\chuhao}{\fontsize{42.2pt}{\baselineskip}\selectfont}
\newcommand{\xiaochu}{\fontsize{36.1pt}{\baselineskip}\selectfont}
\newcommand{\yihao}{\fontsize{26.1pt}{\baselineskip}\selectfont}
\newcommand{\xiaoyi}{\fontsize{24.1pt}{\baselineskip}\selectfont}
\newcommand{\erhao}{\fontsize{22.1pt}{\baselineskip}\selectfont}
\newcommand{\xiaoer}{\fontsize{18.1pt}{\baselineskip}\selectfont}
\newcommand{\sanhao}{\fontsize{16.1pt}{\baselineskip}\selectfont}
\newcommand{\xiaosan}{\fontsize{15.1pt}{\baselineskip}\selectfont}
\newcommand{\sihao}{\fontsize{14.1pt}{\baselineskip}\selectfont}
\newcommand{\xiaosi}{\fontsize{12.1pt}{\baselineskip}\selectfont}
\newcommand{\wuhao}{\fontsize{10.5pt}{\baselineskip}\selectfont}
\newcommand{\xiaowu}{\fontsize{9.0pt}{\baselineskip}\selectfont}
\newcommand{\liuhao}{\fontsize{7.5pt}{\baselineskip}\selectfont}
\newcommand{\xiaoliu}{\fontsize{6.5pt}{\baselineskip}\selectfont}
\newcommand{\qihao}{\fontsize{5.5pt}{\baselineskip}\selectfont}
\newcommand{\bahao}{\fontsize{5.0pt}{\baselineskip}\selectfont}





% 页边距
\geometry{left=2.8cm,right=2.8cm,top=2.25cm,bottom=2.25cm}  % 上下左右边距设置
% \geometry{scale=0.8, centering}  % 比例设置





% default 风格页眉页脚
%%% - empty: 没有页眉页脚
%%% - fancy: 使用 fancyhdr 风格页眉页脚 (需要宏包 fancyhdr)
%%% - plain: 没有页眉, 页脚是居中的页码 (report,article)
%%% - headings: 没有页脚, 页眉是章节名称与页码 (book,ctex)
%%% - myheadings: 没有页脚, 页眉是页码和用户自定义内容
\pagestyle{plain}
% \markright{content}  % 单面文档修改页眉内容
% \markboth{left_content}{right_content}  % 双面文档分别修改左边与右边页眉内容


% fancyhdr 风格页眉页脚
%%% 设置
% \lhead{content}  % 设置页眉左
% \chead{content}  % 设置页眉中
% \rhead{content}  % 设置页眉右
% \lfoot{content}  % 设置页脚左
% \cfoot{content}  % 设置页脚中
% \rfoot{content}  % 设置页脚右
%%% E(偶数页), O(奇数页), L(左), C(中), R(右), H(页眉), F(页脚)
% \fancyhead[place]{content}  % 设置页眉, E/O 与 LCR 组合
% \fancyfoot[place]{content}  % 设置页脚, E/O 与 LCR 组合
% \fancyhf[place]{content}  % 设置页眉页脚, H/F 与 E/O 与 LCR 组合, 没有内容则会清空指定位置的内容
%%% 页眉页脚线
% \renewcommand{\headrulewidth}{0.4pt}  % 页眉线宽度, 默认为 0.4pt
% \renewcommand{\footrulewidth}{0.0pt}  % 页脚线宽度, 默认0pt





% 章节格式
%%% titleformat{标题}{整体格式}{编号}{间隔}{标题}, e.g.:
% \titleformat{\section}{\center\bfseries}{\Large\S\arabic{section}}{2em}{\LARGE}


% 日期格式 (ctex)
% \CTEXoptions[today=small]  % 2014年4月6日, 默认
% \CTEXoptions[today=big]  % 二〇一四年四月六日
% \CTEXoptions[today=old]  % April 6, 2014





% 目录深度
% \setcounter{secnumdepth}{3}  % 设置编号层级, 默认值为3
% \setcounter{tocdepth}{3}  % 设置目录层级, 默认值为3


% 目录名称
% \renewcommand{\contentsname}{Contents}  % 目录名称
% \renewcommand{\listfigurename}{Figure Contents}  % 图目录名称
% \renewcommand{\listtablename}{Table Contents}  % 表目录名称
% \renewcommand{\listoflistingscaption}{Code Contents} % 源码目录名称
% \renewcommand{\refname}{Reference}  % 修改参考文献名称
% \renewcommand{\abstractname}{Abstract}  % 修改摘要名称





% 标题格式
%%% \captionsetup[环境]{参数}
%%% - format:  多行标题显示, plain(default)/hang
%%% - labelformat: 标签编号格式, simple(default)/empty/brace/parens
%%% - labelsep: 标签空格格式, colon(default)/none/period/space/quad/newline/endash
%%% - justified: 对齐方式
%%% - font: 字体格式
%%% - 标题名称: e.g. thetablename


% 标题名称
% \renewcommand{\indexname}{Index}  % 索引
% \renewcommand{\tablename}{Table}  % 表
% \renewcommand{\figurename}{Figure}  % 图
% \renewcommand{\equationname}{Equation} % 公式
% \renewcommand{\listingscaption}{Code} % 源码


% 标题编号与章绑定
% \numberwithin{table}{section} % 表
% \numberwithin{figure}{section} % 图
% \numberwithin{equation}{section}  % 公式
% \numberwithin{listing}{section} % 源码


% 标题编号与章解除绑定
% \counterwithout{table}{section} % 表
% \counterwithout{figure}{section} % 图
% \counterwithout{equation}{section} % 公式
% \counterwithout{listing}{section} % 源码


% 标题编号格式
% \renewcommand{\thetable}{\arabic{section}-\arabic{table}} % 表
% \renewcommand{\thefigure}{\arabic{section}-\arabic{figure}} % 图
% \renewcommand{\theequation}{\arabic{section}-\fnsymbol{equation}} % 公式
% \renewcommand{\thelisting}{\arabic{section}-\arabic{listing}} % 源码





% 引用格式
%%% 页
\newcommand{\reffmttitle}[2]{第 \ref{#1} #2}
\newcommand{\reffmtitem}[2]{#1 \ref{#2}}
\newcommand{\refthename}[1]{\nameref{#1}}
\newcommand{\refpage}[1]{第 \pageref{#1} 页}
%%% 卷(part)
\newcommand{\refpartnum}[1]{\reffmttitle{part:#1}{卷}}
\newcommand{\refpartname}[1]{\refthename{part:#1}}
\newcommand{\refpartpage}[1]{\refpage{part:#1}}
%%% 章(chapter)
\newcommand{\refchapnum}[1]{\reffmttitle{chap:#1}{章}}
\newcommand{\refchapname}[1]{\refthename{chap:#1}}
\newcommand{\refchappage}[1]{\refpage{chap:#1}}
%%% 节(section)
\newcommand{\refsecnum}[1]{\reffmttitle{sec:#1}{节}}
\newcommand{\refsecname}[1]{\refthename{sec:#1}}
\newcommand{\refsecpage}[1]{\refpage{sec:#1}}
%%% 小节(subsection)
\newcommand{\refsubsecnum}[1]{\reffmttitle{subsec:#1}{小节}}
\newcommand{\refsubsecname}[1]{\refthename{subsec:#1}}
\newcommand{\refsubsecpage}[1]{\refpage{subsec:#1}}
%%% 小小节(subsubsection)
\newcommand{\refsubsubsecnum}[1]{\reffmttitle{subsubsec:#1}{小小节}}
\newcommand{\refsubsubsecname}[1]{\refthename{subsubsec:#1}}
\newcommand{\refsubsubsecpage}[1]{\refpage{subsubsec:#1}}
%%% 脚注(footnote)
\newcommand{\reffnnum}[1]{\reffmtitem{注}{fn:#1}}
\newcommand{\reffnname}[1]{\refthename{fn:#1}}
\newcommand{\reffnpage}[1]{\refpage{fn:#1}}
%%% 图(figure)
\newcommand{\reffignum}[1]{\reffmtitem{图}{fig:#1}}
\newcommand{\reffigname}[1]{\refthename{fig:#1}}
\newcommand{\reffigpage}[1]{\refpage{fig:#1}}
%%% 表(table)
\newcommand{\reftabnum}[1]{\reffmtitem{表}{tab:#1}}
\newcommand{\reftabname}[1]{\refthename{tab:#1}}
\newcommand{\reftabpage}[1]{\refpage{tab:#1}}
%%% 项(item)
\newcommand{\refitemnum}[1]{\reffmtitem{项}{item:#1}}
\newcommand{\refitemname}[1]{\refthename{item:#1}}
\newcommand{\refitempage}[1]{\refpage{item:#1}}
%%% 公式(equation)
\newcommand{\refeqnum}[1]{式 (\ref{eq:#1})}
\newcommand{\refeqname}[1]{\refthename{eq:#1}}
\newcommand{\refeqpage}[1]{\refpage{eq:#1}}
%%% 定理(theorem)
\newcommand{\refthmnum}[1]{\reffmtitem{定理}{thm:#1}}
\newcommand{\refthmname}[1]{\refthename{thm:#1}}
\newcommand{\refthmpage}[1]{\refpage{thm:#1}}
%%% 算法(algorithm)
\newcommand{\refalgonum}[1]{\reffmtitem{算法}{algo:#1}}
\newcommand{\refalgoname}[1]{\refthename{algo:#1}}
\newcommand{\refalgopage}[1]{\refpage{algo:#1}}
%%% 源码(code)
\newcommand{\reflstnum}[1]{\reffmtitem{源码}{lst:#1}}
\newcommand{\reflstname}[1]{\refthename{lst:#1}}
\newcommand{\reflstpage}[1]{\refpage{lst:#1}}





% 文献
\addbibresource{~/org/reference/ref.bib} % 文献数据库
\nocite{} % 最小目录





% 色彩设置
%%% 网站
%%%%% \LaTeX Color: http://latexcolor.com/
%%%%% 中国色: http://zhongguose.com/
%%%%% 日本色: http://nipponcolors.com/
%%% 语法
% \definecolor{\name}{gray}{\var}  % 灰度, 0~1
% \definecolor{\name}{rgb}{\var,\var,\var}  % red/green/blue (0~1)
% \definecolor{\name}{RGB}{\var,\var,\var}  % RED/GREEN/BLUE (0~255)
% \definecolor{\name}{HTML}{\var\var\var}  % RED/GREEN/BLUE (00-FF)
% \definecolor{\name}{cmyk}{\var,\var,\var,\var}  % cyan/magenta/yellow/black (0~1)
\definecolor{FENGYEHONG}{RGB}{194,31,48}  % 枫叶红
\definecolor{GANLANSHILV}{RGB}{178,207,135}  % 橄榄石绿
\definecolor{GULAN}{RGB}{26,148,188}  % 钴蓝
\definecolor{HUPOHUANG}{RGB}{254,186,7}  % 琥珀黄
\definecolor{MIZUASAGI}{RGB}{102,186,183}  % 水浅葱
\definecolor{OUCHI}{RGB}{155,144,194}  % 楝
\definecolor{QIANHUI}{RGB}{218,212,203}  % 浅灰
\definecolor{SAKURA}{RGB}{254,223,225}  % 桜
\definecolor{YINHUI}{RGB}{145,128,114}  % 银灰





% hyper 设置
\hypersetup{
    pdfstartview=FitH,  % 缩放大小: Fit(适合页面), FitH(适合宽度), FitV(适合高度)
    CJKbookmarks=true,  % 生成书签
    bookmarksnumbered=true,  % 书签带编号
    bookmarksopen=false,  % 自动打开书签
    colorlinks=false,  % 启用链接颜色
    pdfborder=000,  % 链接边框
    linkcolor=black,  % 内部链接, 黑色
    anchorcolor=FENGYEHONG,  % 锚文本, 枫叶红
    citecolor=GANLANSHILV,  % 文献引用链接, 橄榄石绿
    filecolor=HUPOHUANG,  % 文件链接, 琥珀黄
    menucolor=MIZUASAGI,  % 菜单链接, 水浅葱
    runcolor=SAKURA,  % 运行注释链接, 桜
    urlcolor=OUCHI,  % URL链接, 楝
    % pdftitle={Title},  % 文档标题
    % pdfauthor={Author},  % 文档作者
    % pdfsubject={Subject}, % 文档主题
    % pdfkeywords={Keyword}, %文档关键字
}
